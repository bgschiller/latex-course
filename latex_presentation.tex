\documentclass{beamer}

%\beamertheme{default}
\usecolortheme{seagull}
\usepackage{verbatim}
\usepackage{moreverb}
\usepackage{multicol}
\usepackage{framed}
\usepackage{listings}
\usepackage{amsmath}
\setbeamertemplate{navigation symbols}{}%remove navigation symbols
\lstset{ %
language=TeX,
basicstyle=\scriptsize\ttfamily}

\title{Introduction to \LaTeX}
\author{Brian Schiller}
\date{\today}

\begin{document}

\frame{\titlepage}

\begin{frame}[fragile]
 \frametitle{Introduction}
 \begin{multicols}{2}
 \begin{itemize}
 	\item[$\bullet$]<1-> WYSIWYG vs WYSIWYM
	\item[$\bullet$]<2-> Control structures typed alongside text.
	\item[$\bullet$]<5-> \begin{verbatim}\command[optional]
	{argument} \end{verbatim}
\end{itemize}

\columnbreak

\pause
\pause
\begin{verbatimtab}
\begin{center} 
	The center of the Earth is 
	composed primarily of 
	a nickel-iron alloy.
\end{center}
\end{verbatimtab}

\pause
\begin{center}
\begin{framed}
\begin{minipage}{0.7\columnwidth}
\begin{center} 
	The center of the Earth is composed primarily of a nickel-iron alloy.
\end{center}
\end{minipage}
\end{framed}
\end{center}

\end{multicols}
\end{frame}

\begin{frame}[fragile]
  \frametitle{The Smallest Document}
\begin{multicols}{2}
\begin{itemize}
	\item[]<1-> \verb|\documentclass{article}|
	\item[]<2-> \verb|\begin{document}|
	\item[]<3-> \verb|Hello, World.|
	\item[]<4-> \verb|\end{document}|   
\end{itemize}
\columnbreak

\begin{center}
\pause
\pause
\pause
\pause
\begin{framed}
\begin{minipage}{0.7\columnwidth}
Hello, World.
\vspace{5cm}
\end{minipage}
\end{framed}
\end{center}
\end{multicols}
\end{frame}

\begin{comment}%beginning comment here
\begin{frame}[fragile]
  \frametitle{Environments}

\begin{itemize}
	\item[]<1->Environments begin with \verb|\begin{...}| and end with \verb|\end{...}|.
	\item[]<2-> Examples: 
			\begin{itemize}
			\item[$\bullet$]<3->center, flushleft, flushright
			\item[$\bullet$]<4->comment, tabular, itemize
			\end{itemize}
	\item[]<5-> Most errors come from a forgotten \verb|\end{something}|.
	\item[]<6-> Environments can be nested, but inner environments must be closed before outer environments:
	\begin{multicols}{2}
	Incorrect ordering 
	\begin{verbatimtab}
	\begin{verse}
	\begin{center}
		Ode to road:
		Hi, way.
	\end{verse}
	\end{center}
	\end{verbatimtab}
	
	\columnbreak
	
	Correct ordering 
	\begin{verbatimtab}
	\begin{verse}
	\begin{center}
		Ode to road:
		Hi, way.
	\end{center}
	\end{verse}
	\end{verbatimtab}
	\end{multicols}
\end{itemize}
\end{frame}
\end{comment}%ending comment here

\begin{frame}[fragile]
	\frametitle{The Preamble}
	\vspace{-12pt}
\begin{itemize}
\item[]<1-> \verb|\documentclass[11pt]{article}| \vspace{-12pt}
\item[]<1,3->\verb|%begins paragraphs with an empty line instead of a tab.|
\verb|\usepackage[parfill]{parskip}|

\item[]<1,4->\verb|%creates smaller margins|
\verb|\usepackage[margins=1in]{geometry}|

\item[]<1,5->\verb|%math commands and symbols|
\verb|\usepackage{amsmath, amssymb}|

\item[]<1,6->\verb|% Theorem and proof environments|
\verb|\usepackage{amsthm}|

\item[]<1,7->\verb|%allows for comment blocks and verbatim sections|
\verb|\usepackage{verbatim}|

\item[]<1,8->\verb|%change font to KP serif|
\verb|\usepackage[T1]{fontenc}|
\verb|\usepackage{kpfonts}|

\item[]<1,9->\verb|\title{A Rudimentary Introduction to \LaTeX}|
\verb|\author{Brian Schiller}|
\verb|\date{\today}|
\item[]<1,10->\verb|\begin{document}|
\end{itemize}

\end{frame}

\begin{frame}[fragile]
	\frametitle{Comments}
	\begin{itemize}
	\item[]<1-> The comment character in \LaTeX{} is the percent sign, \%.
	\item[]<2-> Blank out or toggle commands:
		\verb|%\usepackage{fullpage}|
	\item[]<3-> Block comments require the use of the verbatim package.
	\begin{verbatimtab}
	\begin{comment}
		Everything in this section is ignored. 
		
		Nothing between \begin{comment}
		and \end{comment} is typeset.
	\end{comment}
	\end{verbatimtab}
	\item[]<4-> That code contains an error; who know where?
	\end{itemize}
\end{frame}

\begin{frame}[fragile]
	\frametitle{Math Mode}
	Math mode is different from text mode in a few subtle ways:
	\begin{itemize}
		\item[$\bullet$]<1->Spacing in math mode squeezes everything together. Use \verb|\hspace{...}|.
		\item[$\bullet$]<2->Empty lines are prohibited. Use \verb|\vspace{...}|.
		\item[$\bullet$]<3->Some symbols are treated differently:
			\begin{tabular}{c | c | c}
			Input & Text Mode & Math Mode\\
			\hline
			\verb|2<5| & 2<5 & $2<5$\\
			\verb|5>2| & 5>2 & $5>2$\\
			\verb,3|6, & 3|6& $3|6$
			\end{tabular}			
	\end{itemize}
\end{frame}

\begin{frame}[fragile]
	\frametitle{Typing Math}
	\begin{multicols}{2}
		Inline math is for a situation when you want an expression like \(x^2+y^2 \leq 4\) right in the middle of your paragraph. It is typed \verb|\(x^2+y^2 \leq 4\)|, or alternatively, \verb|$x^2+y^2 \leq 4$|.
		
		\columnbreak
		\pause
		Display math is presented separate from your text, like so:
		\[x^2+y^2 \leq 4\]
		It is typed \verb|\[x^2+y^2 \leq 4\]|.
	\end{multicols}
	\begin{comment}
	You can force Display-sized math in an inline environment by prefixing the expression with \verb|\displaystyle|:
	\verb|\( \displaystyle x^2+y^2 \leq 4\)|
	\end{comment}
\end{frame}

\begin{frame}[fragile]
	\frametitle{Superscripts and Subscripts}
	\begin{itemize}
		\item[]<1-> Subscripts are produced by adding \verb|_{subscript}| to the end of an object. \verb| x_{1} + 3x_{2} = 4 | creates \( x_{1} + 3x_{2} = 4 \).
		\item[]<2-> Superscripts are much the same, with \verb|^{superscript}|. \verb|x^{2}+y^{2} \leq 4| creates \(x^2+y^2 \leq 4\).
		\item[]<3-> This works on operators as well: \verb|\lim_{a \to \infty} \, f(x)| is \(\lim_{a \to \infty} \, f(x)\)
		\item[]<4-> But \( a \to \infty \) is not really where it belongs. We can add \verb|\limits| to put it right where we want it: \verb|\lim \limits_{a \to \infty} \, f(x)| gives \(\lim \limits_{a \to \infty} \, f(x)\).
		\item[]<5-> Note that brackets can be left off if the superscript or subscript is only one character. \verb| x_1 + 3x_2 =4| would be the same as above	
	\end{itemize}
\end{frame}

\begin{frame}[fragile]
\frametitle{Fractions}
\begin{multicols}{2}
	Fractions are typed: \verb|\frac{top}{bottom}|
	
	\vspace{36pt}
	%\pause
	\verb|\frac{\frac{1}{4}}{x^2 + y^2}|
	
	\columnbreak
	
	\[ \frac{top}{bottom}\]
	\pause
	\[ \frac{\frac{1}{4}}{x^2 + y^2} \]

\end{multicols}
\end{frame}

\begin{frame}[fragile]
\frametitle{The Align Environment}
\begin{itemize}
	\item[]<1-> Ampersands, \verb|&| mark the alignment point,
	\item[]<1-> Two backslashes separate each line, \verb|\\|.
	\pause
	 \begin{multicols}{2}
	 	\begin{verbatim}
			\begin{align}
				x&=2a+3a\\
				x&=5a
			\end{align}
		\end{verbatim}
	\columnbreak
		\begin{align}
			x&=2a+3a\\
			x&=5a
		\end{align}
	\end{multicols}
	\item[]<3-> To exclude the numbers, use \verb|align*|:
	 \begin{multicols}{2}
	 	\begin{verbatim}
			\begin{align*}
				x&=2a+3a\\
				x&=5a
			\end{align*}
		\end{verbatim}
	\columnbreak
		\begin{align*}
			x&=2a+3a\\
			x&=5a
		\end{align*}
	\end{multicols}
	\item[]<4-> Note: The Align environment enters math mode automatically.
\end{itemize}
\end{frame}

\begin{frame}[fragile]
\frametitle{The Align Environment}

It is often useful to align two columns at once:

	\begin{align*} 
		x&=x \wedge (y \vee z) &&\text{(by distributivity)} \\ 
		&= (x \wedge y) \vee (x \wedge z) && \text{(by condition (M))}\\ 
		&= y \vee z.
	\end{align*}

\pause
\begin{verbatimtab}
	\begin{align*} 
	   x&=x \wedge (y \vee z) &&\text{(by distributivity)} \\ 
	   &= (x \wedge y) \vee (x \wedge z) 
	      && \text{(by condition (M))}\\ 
	   &= y \vee z.
	\end{align*}
\end{verbatimtab}
\end{frame}

\begin{frame}[fragile]
\frametitle{Matrices}
\begin{itemize}
	\item[]<1-> Entries are separated by an ampersand, \verb|&|.
	\item[]<2-> Rows are separated by two backslashes, \verb|\\|.
\end{itemize} 
	\vspace{24pt}
	\pause
	\pause
	\hspace{-12pt} \begin{tabular}{c | c | c | c | c | c}
		input & \verb|matrix| & \verb|bmatrix| & \verb|pmatrix| & \verb|vmatrix| & \verb|Vmatrix| \\
		\hline
		\begin{lstlisting}
			\[
			\begin{...} 
				1&2 \\
				3&4
			\end{...}
			\]
		\end{lstlisting}
		&
		\( \begin{smallmatrix} 1&2 \\
				3&4
			\end{smallmatrix}\)
		&
			\( [\begin{smallmatrix} 1&2 \\
				3&4
			\end{smallmatrix}]\)
		&
			\( (\begin{smallmatrix} 1&2 \\
				3&4
			\end{smallmatrix})\)		
		&
				\( |\begin{smallmatrix} 1&2 \\
				3&4
			\end{smallmatrix}|\)		
		&
				\( ||\begin{smallmatrix} 1&2 \\
				3&4
			\end{smallmatrix}||\)
	\end{tabular}	
\end{frame}

\begin{frame}[fragile]
\frametitle{Theorems and Proofs}
\begin{itemize}
	\item[]<1-> In preamble: \verb|\usepackage{amsthm}|
	\item[]<1->  \hspace{12ex} \verb|\newtheorem{lem}{Lemma}|
	\item[]<2-> Now, anywhere in the document:
		\begin{verbatim}
			\begin{lem}
				If p is prime and $p \mid ab$
				 then $p \mid a$ or $p \mid b$.
			\end{lem}
		\end{verbatim}
	\item[]<3-> For proofs:
		\begin{verbatim}
			\begin{proof}[optional title]
				First, consider the number of primes: at least five...
			\end{proof}
		\end{verbatim}
\end{itemize}
\end{frame}

\begin{frame}[fragile]
\frametitle{Tables}
\begin{itemize}
	\item[]<1-> \protect\verb!\begin{tabular}{c | c | c | c}!
	\item[]<1,9-> \verb|   Var & \multicolumn{3}{c}{Functions}\\|
	\item[]<1,3-> \verb|   $x$ & $2x$ & $x^2$ & $2x^2$ \\|
	\item[]<1,4-> \verb|   \hline|
	\item[]<1,5-> \verb|   -1&-2&1&2\\|
	\item[]<1,6-> \verb|   0&0&0&0\\|
	\item[]<1,7-> \verb|   1 & 2 & 1 & 2\\|
	\item[]<1,8-> \verb|\end{tabular}|
\end{itemize}
\begin{flushright}
\begin{tabular}{c | c | c | c} 
Var & \multicolumn{3}{c}{Functions}\\ 
$x$ & $2x$ & $x^2$ & $2x^2$ \\ 
\hline 
-1&-2&1&2\\ 
0&0&0&0\\ 
1 & 2 & 1 & 2\\
\end{tabular}
\end{flushright}
\end{frame}

\begin{frame}[fragile]
\frametitle{Source Code}

\begin{itemize}
	\item[]<1-> Source Code: include in preamble \verb|\usepackage{listings}|
	\item[]<1-> and also: \verb|\lstset{language=Ada, | \vspace{-14pt}
	\item[]<1-> \verb|        basicstyle=\footnotesize\ttfamily ... etc}|
	\item[]<2-> Then, anywhere in the document:
		\begin{verbatim}
			\begin{lstlsting}
			   while numcopy/=0 loop
			      sum:=sum+ (numcopy mod 10);
			      numcopy:= numcopy/10;
			   end loop;
			\end{lstlisting}
		\end{verbatim}
	\item[]<3->Also available: \verb|verbatim| and \verb|algorithmic|.
\end{itemize}
\end{frame}

\begin{frame}[fragile]
\frametitle{Other Notes}
\begin{itemize}
	\item[]<1-> Quotation marks: don't use \verb|"| (next to enter). You want \verb|`| (under esc) on the left, and \verb|'| (next to enter) on the right.
	\item[]<1-> "Wrong quotes"
	\item[]<1-> ``Proper quotes''
	\item[]<2-> \LaTeX{} is CaSe sEnSiTiVe; \verb|\Begin{document}| won't work.
	\item[]<3-> The \verb|multicol| and \verb|geometry| packages.
	\item[]<4-> Never from a thumbdrive!
	\item[]<5-> \href{http://detexify.kirelabs.org/classify.html}{Detexify}
\end{itemize}
\end{frame}
\end{document}
