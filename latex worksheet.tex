\documentclass{article}
\usepackage{amsmath}
%\usepackage{fullpage}
\usepackage{amsthm}

\usepackage[margin=1in]{geometry}
\usepackage{listings}
\lstset{language=Ada, basicstyle=\ttfamily\small}

\newtheorem*{thm}{Schiller/Whitney Similarity Theorem}

\title{My \LaTeX{} Portfolio}
\author{your name}
\date{September 28, 2011}

\begin{document}
\maketitle
\thispagestyle{empty}
Well, three or four months run along, and it was well into the winter now. I had been to school most all the time and could spell and read and write just a little, and could say the multiplication table up to six times seven is thirty-five, and I don't reckon I could ever get any further than that if I was to live forever.  I don't take no stock in mathematics, anyway. \begin{flushright} {\large  --Huckleberry Finn} \end{flushright}
\verb|Quote attribution typeset with \begin{flushright} {\large --Huckleberry Finn} \end{flushright}|

\[
\int \ln(t) \, \mathrm{d}t= t \ln(t) - \int 1\, \mathrm{d}t 	\qquad \text{(Integration by Parts)}
\]
\verb|Text can be accomplished inside math mode with \text{...}, but only if you load the package amsmath|
\[
\begin{bmatrix}
	1 & 2 & 3 \\
	0 & -6 & 7
\end{bmatrix}^T =
\begin{bmatrix}
	1 & 0 \\
	2 & -6 \\
	3 & 7
\end{bmatrix}
\]
\verb|Those are typeset with \begin{bmatrix}, and the `T' exponent is just a superscript|
\begin{align*}
a&=\sum_{n=0}^\infty (b_n-b_{n+1})\\
&=\lim_{n \to \infty} (b_0 - b_1) + (b_1 -b_2) + \ldots + (b_{n-1} - b_k)\\
&=\lim_{n \to \infty} (b_0-b_n) \hspace{2in} \text{(by Additive Cancellation)}
\end{align*}
\verb|For this, you will need the align* environment. Don't worry about centering until the end.|


\[
\frac{\sqrt{\frac{xy}{1}(\frac{1}{x}+\frac{1}{y})}}{\frac{xy}{2}\frac{1}{xy}}=\frac{\sqrt{y+x}}{\frac{1}{2}}=2\sqrt{x+y}
\]
\vspace{3ex}


The difference quotient of a function $f$ around a point $a$ is defined as \( \displaystyle\lim_{x \to a} \frac{f(x)-f(a)}{x-a} \). Often, in beginning differential calculus, students are required to calculate derivatives this way. This is generally agreed to be a huge pain.
\vspace{2ex}


\verb|The expression in this paragraph is typeset using \displaystyle.|

 \pagebreak
 
\begin{thm} Let $A$ and $B$ be matrices. Then, $A$ and $B$ are similar.
\begin{proof}
Well, if $A$ and $B$ are both matrices, then they are boxes with numbers in them. So, they are similar in that way.
\end{proof}
\end{thm}
\hspace{-0.8in} \verb|You will need the package `amsthm' and you will need to declare a new theorem with \newtheorem in the preamble.|

\begin{center}
\begin{tabular}{ c | c | c | c | c }
\multicolumn{5}{  c }{modulo 4} \\
 $\times$ & 0 & 1 & 2 & 3 \\
	    \hline
 	    0 & 0 & 0 &  0 &0 \\
	    \hline
	    1 & 0 & 1 & 2 & 3 \\
	    \hline
	    2 & 0 & 2 & 0 & 2 \\
	    \hline
	    3 & 0 & 3 & 2 & 1\\
\end{tabular}
\end{center}
\verb|This is the tabular environment, including a multicolumn line.|

\begin{lstlisting}
with Ada.Text_Io; use Ada.Text_Io;
 
procedure Gcd_Test is
   function Gcd (A, B : Integer) return Integer is
      M : Integer := A;
      N : Integer := B;
      T : Integer;
   begin
      while N /= 0 loop
         T := M;
         M := N;
         N := T mod N;
      end loop;
      return M;
   end Gcd;
 
begin
   Put_Line("GCD of 100, 5 is" & Integer'Image(Gcd(100, 5)));
   Put_Line("GCD of 5, 100 is" & Integer'Image(Gcd(5, 100)));
   Put_Line("GCD of 7, 23 is" & Integer'Image(Gcd(7, 23)));
end Gcd_Test;

\end{lstlisting}

\verb!This is the listings package, with the settings language=Ada, basicstyle=\ttfamily\small.!
\verb|The code is available online at Rosetta Code.|
\end{document}
